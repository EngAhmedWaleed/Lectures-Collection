\setchapterpreamble[u]{\margintoc}
\chapter{Second Lecture}
\labch{lec2}

\section[Intro. to Mathematical Models]{Introduction to Mathematical Models}
\labsec{sec2.1}

The purpose of this chapter is to present methods of writing the differntial equations for a variety of electrical and mechanical systems.
This is the first step that must be mastered by the would-be control systems engineer.\\

Series Resistor-Inductor-Capacitor Circuit\\ \\
$V_R=R\ i$\\
$V_L=(LD)\ i$\\
$V_C=(\dfrac{1}{CD})\ i$\\

$v = (LD + R + \dfrac{1}{CD})\ i$\\
$V(s) = (Ls + R + \dfrac{1}{Cs})\ I(s)$\\

R -> Resistor, L -> Inductor, C -> Capacitor\\

Simple Mechanical Translation System\\
$f_M=(MD^2)\ x$\\
$f_B=(BD)\ x$\\
$f_K=K\ x$

$f = (MD^2+BD+K)\ x$\\
$F(s) = (Ms^2+Bs+K)\ X(s)$\\

M -> Mass, B -> Damping or viscous friction, K -> Elastanse, or stiffness\\

Simple Mechanical Rotational System\\
$\tau_M=(JD^2)\ \theta$\\
$\tau_B=(BD)\ \theta$\\
$\tau_K=K\ \theta$

$\tau = (JD^2+BD+K)\ \theta$\\
$T(s) = (Js^2+Bs+K)\ \Theta(s)$\\

J -> Moment of inertia, B -> Damping or viscous friction, K -> Elastanse, or stiffness\\

Single-stage Rotating Amplifier (Field Controlled Generator)\\
$v_f=(DL_f+R_f)\ i_f$\\
$e_g=K_g\ i_f$\\

$v_f=(\dfrac{DL_f+R_f}{K_G})\ e_g$\\

F -> field, G -> Generator

D-C Servomotor (Armature Controlled Motor)\\

\ldots


\begin{description}
\item[Reference] Feedback control system analysis and synthesis.\footnote{Sections 2-2, 2-3, 2-5, 2-9, 2-10 ``2nd Edition''}
			\\\note{John j. Dazzo, Constantine h. Houpis}
\end{description}