\setchapterpreamble[u]{\margintoc}
\chapter{Second Lecture}
\labch{lec2}

\section[Intro. to Mathematical Models]{Introduction to Mathematical Models}
\labsec{sec2.1}
We will be studying single output linear continuous systems. If a system has more than one input
the superposition principle will be applied.


\subsection[Super Position Principle]{Super Position Principle\linkC{https://en.wikipedia.org/wiki/Superposition_principle}}
 The superposition property, states that, for all linear systems, the net response caused by two
 or more stimuli is the sum of the responses that would have been caused by each stimulus individually.\\[+1em]
\hspace*{\fill}\fbox{
		    \parbox{3.8cm}{
		    $\ F(x_1 + x_2) = F(x_1) + F(x_2)$
	    	}
}\\[+1em]
\note{It can be used to prove linearity}


\subsection{Simple Systems Equations}
The purpose of this section is to present methods of writing the differential equations for a variety of electrical and mechanical systems.
This is the first step that must be mastered by the would-be control systems engineer.\\

Series Resistor-Inductor-Capacitor Circuit\\ \\
$v_R = R\ i$\\
$v_L = (LD)\ i$\\
$v_C = (\dfrac{1}{CD})\ i$\\

$v = (LD + R + \dfrac{1}{CD})\ i$\\
$V(s) = (Ls + R + \dfrac{1}{Cs})\ I(s)$\\

R -> Resistor, L -> Inductor, C -> Capacitor\\

Simple Mechanical Translation System\\
$f_M = (MD^2)\ x$\\
$f_B = (BD)\ x$\\
$f_K = K\ x$

$f = (MD^2+BD+K)\ x$\\
$F(s) = (Ms^2+Bs+K)\ X(s)$\\

M -> Mass, B -> Damping or viscous friction, K -> Elastance, or stiffness\\

Simple Mechanical Rotational System\\
$\tau_J = (JD^2)\ \theta$\\
$\tau_B = (BD)\ \theta$\\
$\tau_K = K\ \theta$

$\tau = (JD^2+BD+K)\ \theta$\\
$T(s) = (Js^2+Bs+K)\ \Theta(s)$\\

J -> Moment of inertia, B -> Damping or viscous friction, K -> Elastance, or stiffness\\

Single-stage Rotating Amplifier (Field Controlled Generator)\\
$v_f = (DL_f+R_f)\ i_f$\\
$v_g = K_g\ i_f$\\

$v_f = (\dfrac{DL_f+R_f}{K_G})\ v_g$\\

F -> field, G -> Generator\\

D-C Servomotor (Armature Controlled Motor)\\
$v_m = (K_b\ D)\ \theta_m$\\
$\tau = K_T\ i_m,\ also:\  \tau = (JD^2+BD)\ \theta_m$ \note{Stiffness neglected}\\	%--> �����
$i_m = (\dfrac{JD^2+BD}{K_T})\ \theta_m$\\
$v_a = v_m + (L_m D+R_m)\ i_m $\\

$v_a = [\dfrac{(L_m J)\ D^3 + (L_m B+R_m J)\ D^2 + (R_m B + K_b K_T)\ D}{K_T}]\ \theta_m$

M -> Motor, A -> Armature\\

D-C Servomotor (Field Controlled Motor)\\
$\tau = K_f\ i_f,\ also:\  \tau = (JD^2+BD)\ \theta_m$ \note{Stiffness neglected}\\	%--> �����
$i_f = (\dfrac{JD^2+BD}{K_f})\ \theta_m$\\
$v_f = (DL_f+R_f)\ i_f$\\	%--> �����

$v_f = [\dfrac{(JD^2+BD)(DL_f+R_f)}{K_f}]\ \theta_m$

F -> Field, M -> Motor\\

\begin{description}
\item[Reference] Feedback control system analysis and synthesis.\footnote{Sections 2-2, 2-3, 2-5, 2-9, 2-10 ``2nd Edition''}
			\\\note{John j. Dazzo, Constantine h. Houpis}
\end{description}

\section[Block Diagram Reduction $_{p\ 2}$]{Reduction Techniques (Moving Points)}
\labsec{sec2.2}
can be done easily \ldots\\

\section{Schematic to Block Diagram}
\labsec{sec2.2}
Fig 5-18 page 5 of 11 \ldots\\
