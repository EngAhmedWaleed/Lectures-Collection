\setchapterstyle{lines}
\chapter{Table of Laplace Transforms}
\labch{laplace}

\begin{description}
\item[Remember] that we consider all functions as defined only on $t \geq 0$.
\end{description}

\begin{margintable}[-0.315cm]
\caption[Canonical form of feedback control system]{Terminology}
\labtab{canonicalTable}
\centering
\tiny
 	\begin{tabular}{|c c l|}
	        \hline
	        \multicolumn{3}{c}{}\\[-1em]
	        R &: & reference input / desired output response.\\
	        E &: & actuating / error signal.\\
	        G &: & control element and controlled system.\\
	        C &: & controlled variable / actual output.\\
	        H &: & feedback / backward transfer element.\\
	        B &: & primary feedback.\\
	        s &: & summation point.\\
	        t &: & takeoff point.\\
	        \multicolumn{3}{c}{}\\[-1em]
	        \hline
	    \end{tabular}
\end{margintable}

\begin{table}[h]
\caption[Laplace general transforms]{General}
\labtab{laplaceGeneralTable}
 	\begin{tabular}{|c c l|}
	        \hline
	        \multicolumn{3}{c}{}\\[-1em]
	        R &: & reference input / desired output response.\\
	        E &: & actuating / error signal.\\
	        G &: & control element and controlled system.\\
	        C &: & controlled variable / actual output.\\
	        H &: & feedback / backward transfer element.\\
	        B &: & primary feedback.\\
	        s &: & summation point.\\
	        t &: & takeoff point.\\
	        \multicolumn{3}{c}{}\\[-1em]
	        \hline
	    \end{tabular}
\end{table}

Thanks \ldots
\todo{Provide a link to the file}