\setchapterstyle{lines}
\chapter{Laplace Transforms}
\labch{laplace}

\newcommand{\Laplace}[1]{\ensuremath{\mathcal{L}{\left[#1\right]}}}
\newcommand{\row}[2]{\multicolumn{2}{c}{}\\[-1em] $#1$ & $#2$\\ \multicolumn{2}{c}{}\\[-1em]}
\newcommand{\separation}{
			\multicolumn{2}{c}{}\\[-1em]
	        \hline
	        \multicolumn{2}{c}{}\\[-1em]
}

\begin{description}
\item[Remember] that we consider all functions are defined only on $t \geq 0$.
\end{description}

\begin{margintable}[-0.312cm]
\caption[Laplace transform theorems]{Theorems}
\labtab{laplaceSpecificTable}
\centering
\tiny
 	\begin{tabular}{l | l}
	        \hline
	        \multicolumn{2}{c}{}\\[-1em]
	        $f(t)$ 	& $\Laplace{f(t)}=\displaystyle{\int_0^\infty f(t)\ e^{-st}dt}=F(s)$\\
	        \separation
	        \row{f(at)}{\dfrac{1}{a}F(\dfrac{s}{a})}
	        \row{\grave{f}(t)}{sF(s)-F(0)}
	        \row{\displaystyle{\int_0^t f(x)dx}}{\dfrac{1}{s}F(s)}
	        \separation
	        \row{t f(t)}{-\grave{F}(s)}
	        \row{\dfrac{1}{t}f(t)}{\displaystyle{\int_s^\infty F(x)dx}}
	        \separation
	        \row{e^{at}f(t)}{F(s-a)}
	        \row{f(t-a)\mathcal{U}(t-a)}{e^{-as}F(s)}
	        \separation
	        \row{\displaystyle{\int_0^t f(x)g(t-x)dx}}{F(s)\ G(s)}
	        \multicolumn{2}{c}{}\\[-1em]
	        \hline
	    \end{tabular}
\end{margintable}

\begin{table}[h]
\caption[Laplace general transforms]{General transforms\linkC{http://integral-table.com}}
\labtab{laplaceGeneralTable}
 	\begin{tabular}{l | l}
	        \hline
	        \multicolumn{2}{c}{}\\[-1em]
	        $f(t)$ 	& $\Laplace{f(t)}=\displaystyle{\int_0^\infty f(t)\ e^{-st}dt}=F(s)$\\
	        \separation
	        \row{a}{\dfrac{a}{s}}
	        \row{\delta(t-a)}{e^{-as}}
			\row{\mathcal{U}(t-a)}{\dfrac{1}{s}\ e^{-as}}
			\row{e^{at}}{\dfrac{1}{s-a}}
	        \row{\sin at}{\dfrac{a}{s^2+a^2}}
	        \row{\cos at}{\dfrac{s}{s^2+a^2}}
	        \row{\sinh at}{\dfrac{a}{s^2-a^2}}
	        \row{\cosh at}{\dfrac{s}{s^2-a^2}}
			\multicolumn{2}{c}{}\\[-1em]
	        \row{t^p}{\dfrac{\Gamma(p+1)}{s^{p+1}}\ ,\  _{p>-1}}
	        \multicolumn{2}{c}{}\\[-1em]
	        \hline
	    \end{tabular}
\end{table}

\begin{table}[h]
\caption[Laplace specific transforms]{Specific transforms\linkS{https://drive.google.com/open?id=1bzh4fp2KrQu1WUOI2OGgwk-Pc214ishB}}
\labtab{laplaceGeneralTable}
 	\begin{tabular}{l | l}
	        \hline
	        \multicolumn{2}{c}{}\\[-1em]
	        $f(t)$ 	& $\Laplace{f(t)}=\displaystyle{\int_0^\infty f(t)\ e^{-st}dt}=F(s)$\\
	        \separation
	        \row{\dfrac{d^n}{dt^n}}{s^{\boldsymbol{n}_{ote}\ \rightsquigarrow} \ \footnote{\tiny Despite being not convinced, it's mentioned in the doctor's notes.}}
			\row{e^{-at}\sin \omega t}{\dfrac{\omega}{(s+a)^2+\omega^2}}
			\row{e^{-at}\cos \omega t}{\dfrac{s+a}{(s+a)^2+\omega^2}}
			\row{\displaystyle{\int_{-\infty}^t f(x)dx}}{\dfrac{1}{s}F(s)+\displaystyle{\int_{-\infty}^0 f(x)dx}}
	        \multicolumn{2}{c}{}\\[-1em]
	        \hline
	    \end{tabular}
\end{table}
\note{$a: constant,x: dummy\ variable, p: real\ number, n: integer$}

\begin{description}
\item[Gamma function] which is defined as:\\
	\hspace*{\fill}$\Gamma \left( p \right) = \int\limits_0^\infty {e^{ - x} x^{p - 1} dx}$\linkS{http://equplus.net/eqninfo/Equation-322.html}\ \ \ \ \ \ \ \ \ \ \ \ \ \ \ \ \ \ \ \ \\
	\normalsize If $n$ is a positive integer then,\\
	\hspace*{\fill}$\Gamma \left( p \right) = p!$\ \ \ \ \ \ \ \ \ \ \ \ \ \ \ \ \ \ \ \ \ \ \ \ \ \ \ \\
\end{description}

