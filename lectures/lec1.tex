\setchapterpreamble[u]{\margintoc}
\chapter{First Lecture}
\labch{lec1}

% Divide Graphviz Images Width by 95 pixel then round.

\section{Introduction}
\labsec{sec1.1}

\begin{description}
	\item[Analysis of linear continuous system] analysis of a system means simply check the goodness of its measure of performance.
Analysis could be done in two different ways:
\begin{itemize}
	\item In the lab: by putting test input to the system and check the output how it satisfies the measure of performance.
	\item Using analytical techniques: which is our concern in this course.
\end{itemize}
	\item[The first step] is to make a mathematical model to the system. \\[+1mm]
	$\Sigma F_x = m\ddot{x}$\\
	$F - \mu\dot{x} + kx = m\ddot{x}$\\
	$\therefore$
	\fbox{
	    \parbox{3cm}{
	    $\ F = m\ddot{x} + \mu\dot{x} + kx$
    		}
	}\\[-2mm]
	\item[Then] defining the measure of performance and study how analytically can check these measure of performance.
\end{description}

\begin{marginfigure}[-3cm]
		\includegraphics{lec1/Dynamics}
		\caption[Spring problem: figure]{A block attached to a spring.\link{https://slideplayer.com/slide/677255/}}
		\labfig{dynamicsSystemFigure}
\end{marginfigure}

 \leavevmode\\[-1.55cm]
 \begin{figure}[hb]
		\raggedleft
		\includegraphics[width=0.65\textwidth]{lec1/Better Block Diagram}
		\caption[Spring problem: block diagram]{Since D is an operator (can't have a value),  the transfer function is obtained by the Laplace transform of the first relation.}
		\labfig{dynamicsSystemBlockDiagram}
\end{figure}
 \leavevmode\\[-1.3cm]

\begin{description}
	\item[Transfer function] ratio between Laplace transform of the output and Laplace transform of the input, assuming zero initial conditions.
\end{description}
 \leavevmode\\[-5em]

\section{Control Systems}
\labsec{sec1.2}
A control system is an interconnection of components forming a system configuration that will provide a desired system response.
\\[-2em]

\subsection[Open-loop control system]{Open-loop control system (without feedback):}
\begin{figure}[hb]
		\raggedleft
		\includegraphics[width=0.41\textwidth]{lec1/Open-loop control system}
		\caption[Open-loop: block diagram]{Its output does not track the input, and it is more affected by noise.}
		\labfig{openLoopBlockDiagram}
\end{figure}
 \leavevmode\\[-4em]

\subsection[Closed-loop control system]{Closed-loop feedback control system (with feedback):}

\begin{figure}[hb]
		\raggedleft
		\includegraphics[width=0.8\textwidth]{lec1/Closed-loop control system}
		\caption[Closed-loop: block diagram]{Closed loop control can improve accuracy, also the actuating signal is a function of the output.}
		\labfig{closedLoopBlockDiagram}
\end{figure}

\section{Mathematical Model}
Any linear continuous system can be represented either by a linear algebraic equation of an ordinary differential equation such as:\\[-4mm]

%https://tex.stackexchange.com/questions/195774/how-to-right-align-any-line-or-word-in-a-paragraph-in-any-documentclass
\hspace*{\fill} $(mD^2 + \mu D + k)\ x(t) = y(t)$\\[-4mm]

Solving the differential equation using Laplace transform assuming zero initial conditions made it possible to get the transfer function.

\section{Block Diagram Reduction}

\begin{margintable}[-0.5cm]
\caption[Canonical form: table]{Terminology}
\labtab{canonicalTable}
\centering
\tiny
 	\begin{tabular}{|c c l|}
	        \hline
	        \multicolumn{3}{c}{}\\[-1em]
	        R &: & reference input / desired output response.\\
	        E &: & actuating / error signal.\\
	        G &: & control element and controlled system.\\
	        C &: & controlled variable / actual output.\\
	        H &: & feedback element.\\
	        B &: & primary feedback.\\
	        s &: & summation point.\\
	        t &: & takeoff point.\\
	        \multicolumn{3}{c}{}\\[-1em]
	        \hline
	    \end{tabular}
\end{margintable}

\begin{marginfigure}[-0.5cm]
		\includegraphics{lec1/Canonical Block Diagram}
		\caption[Canonical form: block diagram]{Canonical feedback loop.}
		\labfig{canonicalBlockDiagram}
\end{marginfigure}

Control systems require the arithmetic manipulation in order to obtain the overall transfer function and this is the start point for the analytical analysis of the system.\\

\begin{tabular}{r p{0.1cm} c}
Cascade connection & : &\\
					&&  \includegraphics[width=0.6\textwidth]{lec1/Series Both}\\
Parallel connection & : &\\
					&&  \tiny\ \ \ \includegraphics[width=0.6\textwidth]{lec1/Parallel Both}\\
					&& \\
Summation point & : &\\
					&&  \multicolumn{1}{l}{some definition ...}\\
Take-off point & : &\\
					&&   \multicolumn{1}{l}{some definition ...}\\
\end{tabular}

\subsection[Overall transfer function]{Overall transfer function (feedback loop elimination):}
El Hamd Lelah\ldots